% Preface template:
% \textit{Об’єкт дослідження}:

% \textit{Предмет дослідження}:

% \textit{Метою роботи є}

\textit{Об’єкт дослідження}:
студенти, експертні системи, їх взаємодія.

\textit{Предмет дослідження}:
поведінка студентів при навчанні, реакція на різноманітний зовнішній вплив.
Використання поведінки студентів для розробки експертної системи оцінювання
і класифікації студентів.
Окремий інтерес являє собою правильність трактування даних,
що отримуються від користувачів в результаті взаємодії з експертною системою,
що є важливішою і складнішою частиною роботи.

\textit{Метою роботи є}
побудова системи комп’ютерного навчання студентів на прикладі обробки
геоінформаційних даних.
Така задача є достатньо багатогранною, для її втілення потрібно працювати
з даними різного походження, також природньо виникає потреба використання
розподіленої системи.
На такій базі буде розроблена достатньо загальна система навчання,
що використовуватиметься для інших навчальних програм.

Основна задача --- мінімізувати витрати часу експерта (викладача) на процес
перевірки знань студентів та надання їм навчальних матеріалів,
підвищення об’єктивності оцінювання студентів.

Тут постає питання інформаційної безпеки, адже потрібно контролювати студентів
--- перевіряти, чи та людина сидить за своїм робочим місцем (щоб запобігати
списування), гарантувати максимальну безпеку базі даних з результатами робіт,
стежити за тим, щоб студенти отримували коректні завдання,
а розв’язки відправлялися системі в незміненому вигляді.
Також потрібно стежити за безпекою зв’язку серверів, що оцінюють роботи
студентів, поведінку студентів та інші важливі показники.
