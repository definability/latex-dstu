Одним з найбільш повних джерел за тематикою автоматизації комп’ютерного навчання
є книга Булигіна В.Г. ``Основы автоматизации процесса обучения'' \cite{Bulygin}.
В ній досить повно розглядається проблема автоматизації навчання ---
від математичних моделей окремих компонент до фізичної реалізації цілої
загальнодержавної системи, мета якої --- покращення якості навчання взагалі.
Тим не менш, система, якій присвячено даний звіт, є складною, і необхідні для її
створення знання виходять за межі книги, хоч і переслідує все ту ж мету ---
покращення і автоматизація процессу навчання.
Наприклад, задачі створення рекомендацій для викладача, аналізу практичних робіт
студентів на предмет нечесної здачі тощо, є важливими, але так і не були
висвітлені в книзі.

Після поверхневого огляду задач системи та методів, якими вони повинні
розв’язуватись, було виділено наступні галузі знань, які потрібно більш детально
розібрати для досягнення поставленої мети:

\begin{enumerate}
    \item
        \textbf{Математичні методи прийняття рішень} для багатокритеріальних
        альтернатив.
        Дана галузь охоплює задачу вибору відповідних завдань для
        конкретного студента з урахуванням багатьох критеріїв, кількість яких
        може змінюватись як в рамках конкретного завдання, так і в залежності
        від даних, які є для конкретного студента.
        Наприклад, якщо студент вже виконував задачу $n$, то система повинна
        знайти для нього інше завдання, аби підвисити об’єктивніть оцінювання.

        Основним джерелом за темою прийняття рішень на момент написання звіту
        є курс Смірнова С.А. ``Моделі та методи прийняття рішень'', що дав
        базове бачення предметної області.
    \item
        \textbf{Методи машинного навчання} в реалізації системи необхідні для
        аналізу результатів завдань і створення необхідних рекомендацій щодо
        покращення рівня навчання. Планується аналізувати як користувачів
        (успішність та ступінь чесності студентів, об’єктивність оцінювання
        перевіряючих осіб), так і складність самих тестових завдань,
        щоб зробити оцінювання якомога більш об’єктивним.

        Для ознайомлення з деякими методами були використані матеріали ІКД
        для внутришнього користування та інформація з сайту MachineLearning.Ru
        \cite{Machinelearning}.
        Дана тема потребує більш детального вивчення перед тим, як будуть
        обрані конкретні методи та алгоритми машинного навчання.

        Під час проходження практики особливу увагу було приділено алгоритмам
        кластерізації k-means та k-means++, в задачі обробки супутникових
        знімків. Були використані як вже готові готові реалізації даних
        алгоритмів в середовищі MatLab, так і окремі реалізації, розроблені
        спеціально для задачі обробки супутникових знімків.
    \item
        \textbf{Безпека розподілених систем} займає важливе місце в даній
        роботі, адже сама система є розподіленою і передбачає використання
        декількох серверів для залучення різноманітних тестових даних.
        Також серед студентів можуть бути особи, що можуть спробувати нечесно
        змінити свої результати, знищити дані або систему, що змусить
        витратити додатковий час на відновлення системи, заново проводити
        тестування, а в крайньому разі унеможливить подальше використання
        системи за призначенням.
    \item
        \textbf{Математична статистика} грає важливу роль у створенні моделей
        студентів та викладачів.
        Система отримуватиме результати роботи користувачів, аналізуватиме їх в
        купі та буде розподіляти їх за певними крітеріями.
        Коли статистичні методи не будуть допомагати, потрібно буде прибігати
        до нейронних мереж для вирішення задачі класифікації.
\end{enumerate}
