Основа реалізації проекту --- розподілена експертна система.
Зрозуміло, що для цього потрібно використовувати мови програмування,
які дозволяють зручно працювати з розподіленими системами,
а також мають якомога кращу підтримку експертних систем --- готові фреймворки,
бібліотеки.

Увагу зачепили дві мови --- Erlang і Scala, коротке представлення про які
наведено далі.

\section{Erlang}
Erlang --- перевірена часом мова програмування, що була створена у 1986 році,
і розвивається до наших днів.
Створений компанією Ericsson спеціально для розподілених систем з великим
навантаженням.
Якщо в інших мовах для підтримки багатопотоковості потрібно завантаження
спеціальних бібліотек, Erlang має ці можливості як основні --- на рівні мови.

Основа розподіленої роботи в Erlang --- обмін повідомленнями між програмами, що
дозволяє розподіляти обробку даних як між різними процесами в межах одного
комп’ютеру, так між різними серверами, полегшується обробка виключних ситуацій
тощо.

Одним з фреймворків для створення експертних систем є ERESYE --- ERlang Expert
SYstem Engine (движок для створення експертних систем на Erlang) \cite{ERESYE}.

ERESYE --- це бібліотека для створення, виконання та управління машинами
виведення \cite{ErlangES}, що й треба для створення експертних систем.

\section{Scala}
Одним з варіантів платформи реалізації описуваної системи є мова програмування
Scala і відповідно віртуальна машина JVM.
Переваги використання цього варіанту:

\begin{enumerate}
    \item 
        Великий вибір готових бібліотек та можливість виконуватись будь-де (дану
        можливість реалізує платформа JVM).
    \item
        Широкий вибір програмного інструментарію для розробки розподілених
        програм.
    \item
        Наявність багатьох готових фреймворків для реалізації експертних систем,
        серед яких:
        \begin{enumerate}
            \item Drools \cite{Drools} --- перевірена часом бібліотека, що має
                відкритий програмний код та велику спільноту розробників,
                що дозволяє швидко знайти відповідь на будь-яке питання її
                стосовно функціональності.
            \item
                D3web \cite{D3web} --- більш проста бібліотека зі своїм
                Wiki-ресурсом та наявністю юніт-тестування.
                Більш просте рішення, яке, можливо, краще підходить для рішення
                задачі, описаної в даному звіті.
            \item
                jColibri \cite{jColibri} --- бібліотека, яка призначена для
                інтерактивного пошуку по великому числу варіантів.
        \end{enumerate}
\end{enumerate}

З вищенаписаного можна зробити висновок, що Scala чудово підходить для
реалізації ідей, описаних в наступному підрозділі.
