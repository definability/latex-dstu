%Машинне навчання\newglossaryentry{computer}

\newglossaryentry{abstraction}{
name={Абстракція},
description={узагальнення більш простих понять до більш
складних, розглядання конкретного явища замість видів, в яких воно може
поставати}
}

\newglossaryentry{machinelearning}{
name={Машинне навчання},
description={підрозділ штучного інтелекту, що вивчає методи побудови моделей,
що здатні до самонавчання}
}

\newglossaryentry{computerstudy}{
name={Комп’ютерне навчання},
description={Навчання людей за допомогою комп’ютера}
}

\newglossaryentry{expertsystem}{
name={Експертна система},
description={Комп’ютерна система, що здатна частково замінити експерта}
}

\newcommand{\textgreek}[1]{\begingroup\fontencoding{LGR}\selectfont#1\endgroup}
\newglossaryentry{ergonomics}{
name={Ергономіка},
description={(від давньогрецького \textgreek{'ergos} --- праця і
\textgreek{n'omos} --- закон) наука, про пристосування робочого місця
(зокрема комп’ютерного) для забезпечення найбільшого комфорту, ефективності
і безпеки}
}

\newglossaryentry{precedence}{
name={Прецедент},
description={специфікація послідовності дій при проектуванні програмних систем}
}

\newglossaryentry{bpmn}{
name={BPMN},
description={(з англійської Business Process Model and Notation, нотація та
модель бізнес-процесів) система умовних позначень для моделювання
бізнес-процесів}
}

\newglossaryentry{}{
name={},
description={}
}

%\newglossaryentry{}{
%name={},
%description={}
%}
