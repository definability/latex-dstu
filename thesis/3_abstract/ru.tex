\chapter*{Реферат}
\thispagestyle{empty}

Неотъемлемой частью оценки успеваемости студентов является анализ результатов
тестирования, однако существующие на данный момент системы тестирования
учитывают лишь правильность ответов.

Цель данной работы --- разработка математической модели психологических
особенностей человека на основе результатов тестирования.
Это поможет создать тестирующую систему, способную давать нужные советы
людям, которые прошли тестирование.

Для достижения поставленной цели
\begin{itemize}
  \item 
    был использован метод главных компонент,
    чтобы извлечь наиболее важные данные из анализируемых выборок;
  \item
    был использован критерий Пирсона ($\chi^2$) для проверки гипотез о
    распределении выборок.
\end{itemize}
