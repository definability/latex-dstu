\chapter*{Вступ}
\addcontentsline{toc}{chapter}{Вступ}

\totalfigures

Значною мірою складова роботи, що стосується психології та педагогіки, завдячує
старшому викладачеві кафедри психології і педагогіки
факультету соціології і права НТУУ ``КПІ'' Апішевій Амінет Шабановні.

\textbf{Актуальність роботи.}
Аналіз результатів тестування є невід’ємною частиною перевірки рівня знань
студентів, проте існуючі на даний момент системи тестування враховують
лише вірність відповідей.
У викладачів може не вистачати часу на докладне знайомство з кожним студентом,
але якщо буде система, що підкаже викладачам, до яких
студентів який підхід краще мати, це безумовно підвищить якість навчання.

\textit{Об’єкт дослідження} ---
студенти.

\textit{Предмет дослідження} ---
психологічні особливості студентів, їх поведінка.

\textbf{Мета дослідження.}
Покращення якості навчання за допомогою порад студентам і викладачам
практичних занять.

Завдання наступні:
\begin{enumerate}
  \item
    Вивчити математичні методи та розділи психології, що дозволять розробити
    математичну модель психологічних характеристик людини,
    пояснити та обґрунтувати отримані результати;
  \item
    Розробити математичну модель людини, що складає тести;
  \item
    Розробити та проаналізувати модель складання тестів людьми різних типів.
\end{enumerate}

\textbf{Практичне значення одержаних результатів.}
Отримані результати
вказують напрям подальших досліджень для створення системи тестування,
яка враховує психологічні особливості людини та надає відповідні поради
для підвищення якості навчання.
