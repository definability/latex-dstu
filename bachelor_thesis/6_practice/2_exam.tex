\section{Моделювання результатів виконання тестових завдань}

\subsection{Інформаційний метод прогнозування часу розв’язання завдань}

Один з методів прогнозування часу розв’язання завдання людиною,
що використовується в інженерній психології, є інформаційний метод.
Вважаємо, що студент виконує завдання одне за одним.
Кожне завдання має певну складність $H_i$ --- кількість інформації,
що потрібно обробити для розв’язку завдання.
Студент виконує кожне завдання з певною швидкістю $V_i$.
Тоді час виконання тесту $\tau$ буде рахуватися за формулою
\begin{equation*}
  \tau = \sum_{i=1}^{l}\frac{H_i}{V_i},
\end{equation*}
де $l$ --- кількість задач. \cite{Lomov:1982}

Вважаємо, що швидкість виконання завдання $i$ рахується за формулою
\begin{equation*}
  V_i\left( t \right) = \frac{1}{\tau_i},
\end{equation*}
де $\tau_i$ --- випадкова величина, що має показниковий розподіл
з параметром $\lambda_i$, який залежить від типу вищої нервової діяльності
людини та часу, що затрачено на виконання попередніх завдань:
\begin{equation*}
  \tau_i \sim Exp\left( \lambda_i \right).
\end{equation*}
Тоді $\tau_i$ буде часом, що витрачається на одиницю інформації
завдання
\begin{equation*}
  \tau = \sum_{i=1}^{l} H_i \cdot \tau_i.
\end{equation*}
Якщо всі завдання мають однакову складність $H_i = H$, рівність приймає вигляд
\begin{equation*}
  \tau = H \cdot \sum_{i=1}^{l} \tau_i.
\end{equation*}

Щоб застосувати отримані з моделювання теппінг-тесту результаті,
потрібно горизонтально масштабувати параболи, до яких було аппроксимовано
ламані.
Початкова парабола $y$ має вигляд
\begin{equation*}
  y \left( t \right) = a^2 \cdot t + b \cdot t + c.
\end{equation*}
Потрібно отримати параболу $\lambda$
\begin{equation*}
  \lambda \left( t \right) = A^2 \cdot t + B \cdot t + C
\end{equation*}
таку, для якої виконується
\begin{equation*}
  \begin{split}
    \lambda\left( 0 \right) &= y\left( 0 \right), \\
    \lambda\left( T \right) &= y\left( t_t \right), \\
    \lambda\left( \frac{T}{2} \right) &= y \left( \frac{t_t}{2} \right)
  \end{split}
\end{equation*}
$t_t$ --- час виконання теппінг-тесту (30 секунд),
$T$ --- час виконання тестових завдань (40 або 90 хвилин).
Маємо розв’язок
\begin{equation*}
  \begin{split}
    A &= a \cdot \left( \frac{t_t}{T} \right)^2, \\
    B &= b \cdot \frac{t_t}{T}, \\
    C &= c,
  \end{split}
\end{equation*}
тобто
\begin{equation*}
  \lambda\left( t \right)
  = a \cdot \left( \frac{t_t}{T} \right)^2 \cdot t^2
    + b \cdot \frac{t_t}{T} \cdot t
    + c.
\end{equation*}

Отже маємо
\begin{equation*}
  \begin{split}
    \tau_1 &\sim Exp\left\{ \lambda\left( 0 \right) \right\}, \\
    \tau_i &\sim
      Exp\left\{ \lambda\left( \sum_{j=1}^{i} H_j \cdot \tau_j \right) \right\},
      \qquad i>1.
  \end{split}
\end{equation*}

\subsection{Моделювання тесту}

Було змодельовано тестування на півтори години з $10$ задачами,
кожна з яких має складність $H=0.5$.
Номер задачі, яку виконує студент на даний момент, фиксується кожні $10$
секунд.
Для застосування методу головних компонент
було змодельовано проходження тестування $600$ студентами.

На рис. \ref{fig:pca:main} можна побачити середньоквадратичні відхилення
перших десяти головних компонент.
Майже половину інформації ($\firstComponentProportion\%$) можна отримати
з першої компоненти.
\begin{figure}[h]
  \centering
    \includegraphics{images/pca}
  \caption{Дисперсії першого десятку головних компонент}
  \label{fig:pca:main}
\end{figure}

На рис. \ref{fig:pca:histograms} наведено гістограми перших чотирьох
головних компонент.
\begin{figure}[h]
  \centering
    \includegraphics{images/pca_hists}
  \caption{Гістограми першої четвірки головних компонент}
  \label{fig:pca:histograms}
\end{figure}
Але з рис. \ref{fig:pca:histograms:detailed} видно, що доцільно
використовувати лише першу компоненту, а також не вдасться відрізнити
від інших класів змішаний та неврівноважений тип.
\begin{figure}[h]
  \centering
    \includegraphics[width=\textwidth]{images/pca_hists_detailed}
  \caption{Детализовані гістограми першої четвірки головних компонент}
  \label{fig:pca:histograms:detailed}
\end{figure}

На рис. \ref{fig:pca1} зображено склад першої головної компоненти
з усіма класами та без змішаного і неврівноваженого типів.
\begin{figure}[h]
  \centering
    \includegraphics[width=0.7\textwidth]{images/pca1_hist}
  \caption{Деталізовані гістограми першої головної компоненти}
  \label{fig:pca1}
\end{figure}

За допомогою алгоритму CART (Classification and Regression Tree) було
класифіковано дві групи студентів:
\begin{enumerate}
  \item
    Ті, кому треба більше часу думати і не поспішати при виконанні завдань,
    щоб запобігти помилок --- рухливий клас, а також не класифіковані
    студенти, адже в основному не вдалося віднести до того чи іншого типу
    тих студентів, активність яких не спадає з часом.
  \item
    Ті, кому треба більше уваги приділяти елементарним задачам,
    щоб довести до автоматизму їх розв’язання і не витрачати час на
    зайві дії --- це студенти, що відносяться до слабкого та інертного типу.
\end{enumerate}
Дерево класифікації зображено на рис. \ref{fig:tree:simple}.
Як можна побачити, критерій досить простий:
\begin{itemize}
  \item
    ті студенти, перша головна компонента яких перевищує поріг,
    відносяться з ймовірністю $\thinkersPredictionQuality\%$ до тих,
    яким треба більше тренуватися;
  \item
    ті студенти, перша головна компонента яких не перевищує поріг,
    відносяться з ймовірністю $\learnersPredictionQuality\%$ до тих,
    яким треба більше думати.
\end{itemize}
\begin{figure}[h]
  \centering
    \includegraphics[width=0.6\textwidth]{images/tree}
  \caption{Дерево класифікації}
  \label{fig:tree:simple}
\end{figure}

Для перевірки результатів на поточній матриці повороту було класифіковано
за першою компонентою ще вибірки по $600$ студентів.
В цих вибірках було вірно класифіковано людей, яким треба більше думати,
з наступною точністю: \testThinkersPredictionQuality.
Тих, яким треба більше працювати, тех було класифіковано з достатньою
точністю: \testLearnersPredictionQuality.
На рис. \ref{fig:tree:test} зображено гістограми першої компоненти
для кожної виборки.

\begin{figure}[h]
  \centering
    \includegraphics[width=\textwidth]{images/pca_hists_test}
  \caption{Інші виборки}
  \label{fig:tree:test}
\end{figure}

\chapterConclusion

В даному розділі за допомогою крітерію Пірсона $\chi^2$ було синтезовано
параметри генератора моделей студентів, за допомогою якого було змодельовано
студентів різних типів, а саме змішаного, інертного, неврівноваженого,
рухливого та слабкого.
Також виявилися студенти, яких не вдалося віднести до жодної групи,
але їх виявилося не так багато, щоб це негативно впливало на якість виборки.

Маючи додатковими знаннями з інженерної психології та експертну оцінку
в області психології вдалося побудувати модель проходження тестування
людьми різних типів.
Результати було проаналізовано за допомогою метода головних компонент.
В ході аналізу виявилося, що найбільшу кількість корисної інформації
містить лише перша компонента, за допомогою якої вдалося з достатньою
точністю надати вірні поради студентам.
Не вдалося знайти особливості студентів, що належать змішаному та
неврівноваженому типів, які б дозволяли виявляти їх з достатньою точністю,
при цьому не погіршуючи результати для інших типів.

На отриманій матриці повороту та дереві класифікації було проаналізовано
ще кілька вибірок.
Аналіз показав, що перша компонента майже не змінює свого розподілу, що
дозволяє і наладі класифікувати студентів з достатньою точністю.
