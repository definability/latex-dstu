\section{Дані з психології}

\subsection{Типи вищої нервової діяльності}

Для визначення того, які показники вимірювати і яким чином,
скористуємось відомою класифікацією типів вищої нервової діяльності.

Згідно з Павловим\cite{Pavlov:1973} типи вищої нервової діяльності
характеризуються трьома показниками: сила нервової системи (сильна або слабка),
врівноваженість (врівноважена або неврівноважена)
та рухливість (рухлива або інетртна).
Павлов розглядає 4 комбінації цих показників з 8 можливих:
\begin{enumerate}
  \item Слабка
  \item Сильна та неврівноважена
  \item Сильна, врівноважена та інертна
  \item Сильна, врівноважена та рухлива
\end{enumerate}
Далі ці класи (комбінації) будуть називатися відповідно слабкий,
неврівноважений, інертний та рухливий.

\subsection{Теппінг-тест (Tapping rate)}
\label{ss:tappingTest}

Існують відомі залежності між типом вищої нервової діяльності та зміною
максимального темпу рухів кистю руки з часом.
Протягом 30 секунд людина намагається притримуватися максимально можливого для
себе темпу.
Показники темпу фіксуються через кожні 5 секунд, а далі по 6 отриманим точкам
будується крива темпа руху. \cite{Ilin:2001}

Для тесту можна використовувати ручку (олівець) і папір, або телеграф.
Сучасні технології дозволяють проводити тест за допомогою клавіатури комп’ютера
або екрану планшета.

З олівцем і папіром тест поводиться наступним чином:
\begin{enumerate}
  \item На папері креслиться 6 квадратів
  \item Людина починає ставити якомога більше точок в першому квадраті впродовж
    перших 5 секунд
  \item Коли проходить 5 секунд, потрібно перейти до наступного квадрату і
    ставити точки там
  \item Процедура повторюється до тих пір, доки не пройде 30 секунд --- в кінці
    буде заповнено всі 6 квадратів
\end{enumerate}
Далі підраховується кількість точок в кожному квадраті та малюється ламана, де
горизонтальна вісь відповідає номеру часового проміжку (номеру квадрата), а
вертикальна відповідає кількості точок в квадраті.

Трактуються отримані дані наступним чином:
\begin{enumerate}
  \item Спадна ламана відповідає слабкому типу (рис. \ref{fig:tapping:weak}).
    Вона спадає після перших 5 секунд тесту і не повертається до початкового
    рівня
  \item Ламана, що спочатку зростає, а після 10-15 секунд спадає нижче
    початкового рівня (проміжна між рівною та опуклою) відповідає
    неврівноваженому типу (рис. \ref{fig:tapping:middle}).
  \item Опукла вниз ламана відповідає інертному типу
    (рис. \ref{fig:tapping:concave}).
    Вона спочатку спадає, а на 25-30 секундах може зрости до початкового темпу
  \item Опукла вгору ламана відповідає рухливому типу
    (рис. \ref{fig:tapping:movable}).
    Це така ламана, що зростає в перші 10-15 секунд тесту, а після 25-30
    секунд повертається або падає нижче початкового рівня
  \item Також темп може залишатися приблизно на одному рівні протягом
    усього тесту, що є оптимальним для складання іспитів
    (рис. \ref{fig:tapping:flat}).
\end{enumerate}

Достовірним відхиленням достатньо вважати різницю в два і більше натиснення між
двома сусідніми п’ятисекундними проміжками. \cite{Ilin:2001}

Оскільки цей тест заснований на вимірюванні витривалості нервової системи людини
за умови максимального навантаження та перевіряє темп реагування (натиснення) на
подразнювачі (внутрішній подразнювач --- команда собі ``треба тиснути''),
було вирішено використовувати відомі вигляди кривих
(рис. \ref{fig:studentBehaviorSimple}) при моделюванні результатів
виконання завдань однакової складності.

\begin{figure}[h]
  \centering
  \begin{subfigure}[b]{0.3\textwidth}
  \resizebox{\textwidth}{!}{\drawHist{chartMovable}}
                \caption{Опукла вгору}
                \label{fig:tapping:movable}
  \end{subfigure}
  \begin{subfigure}[b]{0.3\textwidth}
  \resizebox{\textwidth}{!}{\drawHist{chartMiddle}}
                \caption{Проміжна}
                \label{fig:tapping:middle}
  \end{subfigure}
  \begin{subfigure}[b]{0.3\textwidth}
  \resizebox{\textwidth}{!}{\drawHist{chartFlat}}
                \caption{Рівна}
                \label{fig:tapping:flat}
  \end{subfigure}\\[2ex]
  \begin{subfigure}[b]{0.3\textwidth}
  \resizebox{\textwidth}{!}{\drawHist{chartConcave}}
                \caption{Опукла вниз}
                \label{fig:tapping:concave}
  \end{subfigure}
  \begin{subfigure}[b]{0.3\textwidth}
  \resizebox{\textwidth}{!}{\drawHist{chartWeak}}
                \caption{Спадна}
                \label{fig:tapping:weak}
  \end{subfigure}
  \caption{Загальний вигляд залежностей кількості поставлених точок від часу.
  Пунктирна лінія --- кількість точок в перші 5 секунд}
  \label{fig:studentBehaviorSimple}
\end{figure}

Потрібно зауважити, що швидкість розв’язування задач може змінюватися з
досвідом.
Тобто, якщо студент зі слабкою нервовою системою буде тренуватися виконувати
завдяння, то його показники з часом перейдуть на якісно новий рівень.
Щодо студентів з сильною нервовою системою: швидкість не завжди означає якість
виконання завдань.

Завдання системи --- класифікувати студентів в автоматичному режимі,
з метою подальшого надання порад викладачам практичних занять щодо підвищення
продуктивності роботи кожного студента.
Це полегшує знаходження індивідуального підходу.
Наприклад, тим, хто надто швидко втомлюється, потрібно розв’яювати якомога
більше базових завдань, що не є складними, але розв’язок яких повинен бути
на рівні рефлексів.
Студентам, які поспішають, буде корисно ретельно коментувати у письмовій
формі хід своїх думок, щоб вгамуватися та підвищити свою уважність.

Для тесту було обрано саме 30-секундний проміжок часу, адже спочатку
виміри виконувалися протягом однієї хвилини і було виявлено, що найважливіша
інформація отримується протягом перших 20-25 секунд, а далі лише марно
втрачається час та сили тестованого. \cite{Ilin:2001}

